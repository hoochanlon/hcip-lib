%!TEX program = xelatex
%!TEX TS-program = xelatex
%!TEX encoding = UTF-8 Unicode

\documentclass[12pt, a4paper]{article} % A4 纸,字体大小为 12pt 的 article 类文档
\usepackage{CJKutf8} % 中文支持
\usepackage{graphicx} % 插入图片
\usepackage{subfigure} % 插入多图时用子图显示的宏包
\usepackage{listings} % 支持代码显示
\usepackage[colorlinks,linkcolor=blue]{hyperref} % 超链接
\usepackage{ulem} % 删除线
\usepackage{xcolor} % 定制颜色
\usepackage{caption2} % 浮动图形和表格标题样式
\usepackage{amssymb} % 数学符号
\usepackage{indentfirst} % 中文段落首行缩进
\usepackage{tikz} % 画图
\usepackage{pgfplots} % 画图
\usepackage{amsmath} % 处理数学公式
\usepackage{mathtools} % 处理数学公式
\setlength{\parskip}{0.5em} % 段落间距
\renewcommand{\figurename}{图} % 将图表的标题设置为中文“图”
\usetikzlibrary{tikzmark,calc,decorations.pathreplacing} % tikzmark 用于标记位置,calc 用于计算,decorations.pathreplacing 用于画大括号


\title{第二十四·总结:博弈论的思维方式:人是超级理性的博弈方?}
\author{hoochanlon}
\date{\today}

\begin{document}
	\begin{CJK*}{UTF8}{gbsn}
		\maketitle
        \clearpage
        \section{人生是永不停歇的博弈过程}
        博弈论选择合适策略达到博弈结果,作为博弈者最佳策略是最大限度的利用游戏规则,社会的最佳策略是通过规则引导社会福利整体的增加。
        学习博弈论我们需要从两个角度思考问题,从个体参与的角度,参与博弈的目的就是为了实现自身的利益最大化,怎么制定有效的博弈策略,
        怎么利用好博弈的规则,怎么在竞争中获胜,怎么在合作中共赢,都是作为个体需要考虑的;而从社会的整体视角来看,我们必须站在更高
        层次来看问题,让我们去思考怎么样的游戏规则,才能让社会成员说真话、不偷懒,在规则指引下,让每个人在追求自身利益最大化的同时,
        整个社会整体福利水平也能够不断地增加。

        \subsection{应急管理}
        牢记“预则立不预则废”的应急理念,学会做好各种各样的应急预案。
        \begin{enumerate}
            \item 完整又清晰的信息分类,不能有遗漏。
            \item 责任到人的明确规定,遇到情况不能没有负责人。
            \item 提前准备好详细又可行的行动方案,不要临时抱佛脚。
            \item 时效性,很多紧急情况下,我们需要明确每一项动作的时候要求。
        \end{enumerate}

        \subsection{Baron Lamm Technique}
        \begin{itemize}
            \item 提前几周的准备。
            \item 首创“casing”工作法:去银行踩点,画图,了解银行内部运作。
            \item 给每个人进行明确又细致的分工。
            \item 事先找一个仓库进行反复演练,达到足够熟练水平。
            \item 严格每项流程的执行时间和总时间,规定的时间一到,不管拿没拿到钱都必须马上离开。
            \item 事先侦查并确定好不同天气情况下的跳跑路线,并详细计算好逃跑时间。
            \item 在车的仪表盘粘贴精确到十分之一英里的详细地图。
        \end{itemize}
        好的策略一定是建立在对于可预见事实的洞察和不可见变数的预测之上,我们往往面临很多问题,很多事实,很多信息,但是对这些现象之间的联系,
        以及何为关键联系,决定了能否精炼出有效的策略。我们需要对未来最有可能变化的元素,以及影响变化的条件因素的趋向,进行预测与模拟。如果
        把博弈理解为是人与人之间的比拼,那么运筹帷幄的能力,决定了千里之外的胜负。

        \subsection{唐山烧烤店}
        当自己不清楚对方行为模式,或反应函数之前的以下对策。

        \begin{enumerate}
            \item 尽量避免发生正面冲突是自己占优策略。
            \item 如果冲突发生了,尽量避免暴力升级是你的占优策略。
            \item 当面临人身或财产的严重危险时,尽快寻求警方帮助是自己的占优策略。
        \end{enumerate}

        我们国家的立法精神是当冲突发生时,每个公民是有退让义务的,退让义务几乎贯穿我国所有的制度建设和法律条文。至于以暴制暴,它仅限于严重危及人身安全,
        别人打你不构成你打他的正当理由,所做所为必须是消减和减少社会的暴力总量。谁持械谁理亏,不持械的一方会触发正当防卫。借助公权力对违法犯罪行为进行处罚。\par

        万元损失利用了人性的两个弱点,缺乏远见与厌恶损失。厌恶损失与害怕失去,那是因为,我们想象不到更美好的未来,如果我们在博弈的时候,心中始终有一个更美好的未来,
        你就不会计较眼前的那点得失。如果心中有个更大的目标,着眼于长眼利益,那么我们在和别人发生冲突的时候,就更容易控制自己内心的冲动,更容易采取心平气和的方式解决问题。\par

        \textbf{人在博弈中,首先考虑是是否参与该博弈,然后才是如何在博弈中如何实现利益最大化。其实很多时候,恰恰是不参与博弈才能实现自身利益最大化。
        在博弈之前,博弈就已经开始了。} 因此选对博弈,比选对策略更重要。因此我们在与他人博弈之前,我们胜算的可能性到底有多大,以及参与博弈的期望收益,
        是否明显大于不参与的期望收益。

        \clearpage
        \section{如何在竞争博弈中实现利益最大化}
        博弈论是研究人与人之间的互动关系的。那么人与人之间的关系,要么是竞争,要么是合作。竞争往往是一个此消彼长的关系,合作往往是一个荣辱与共的关系。
        竞争是为了争夺有限的稀缺资源,在竞争博弈中参与者会面临三个具体问题:

        \begin{enumerate}
            \item 比什么
            \item 和谁比
            \item 怎么比
        \end{enumerate}

        很多人很讨厌竞争、害怕竞争,希望能避免竞争,但现实是只要资源稀缺,竞争就不可避免,改变的只是规则、对手和策略。游戏的规则决定了我们未来能力的发展
        方向:
        \begin{itemize}
            \item 如果你有极大的力量,我的地盘我做主制定游戏规则。
            \item 如果你有较大的力量,选择适合你的游戏规则。
            \item 当你缺乏足够的力量,只能去适应游戏规则。(适应不了的结果,就是自然被淘汰)
        \end{itemize}

        我们怎样在竞争博弈中实现自己的利益最大化?我们需要从三个方面来思考问题:

        \begin{enumerate}
            \item 扬长避短
            \item 聚焦聚能
            \item 以强胜弱
            \item 及时止损
        \end{enumerate}

        \clearpage
        \section{如何在合作博弈中实现利益最大化}
        存量思维聚焦如何提高我们的竞争力上面,增量思维的参与者更容易看到合作关系。在人类发展早期生产力极度落后,物质财富极度匮乏,所以很多时候竞争力
        决定了个体和种群的生存能力。随着社会生产力的发展和物质财富的增加,合作能力决定了个体和种群的发展空间。\par

        在现代社会中合作能力比竞争力更重要的能力,那么我们如何在与他人关系中实现自身利益最大化呢?以下几点供参考:

        \begin{enumerate}
            \item 贡献越大,收益越大。
            \item 机会越多,收益越大。
            \item 沟通越容易,收益越大。
            \item 做事越稳,收益越大。
        \end{enumerate}

        竞争博弈:出其不意,攻其不备;合作博弈:言而有信,行而可期。在竞争中,多数时候可能需要隐藏你的行为模式和策略选择,但是在合作中,你尽可能公开,
        需要让对方知道你的行为模式和策略选择。城府深在竞争博弈中,往往对参与者有利。而城府深在合作博弈中,对参与者反而是不利的。《友者生存:与人为善的进化力量》
        我们的人生价值不在于征服了多少敌人,而在于我们交了多少朋友,这就是我们生存的秘密。\par

        博弈的结果源于参与者的策略选择,而参与者的策略选择,往往取决于博弈的规则。一个人善恶可能是源于道德水平的差异,群体的善恶只能是因为社会的游戏规则。
        希望大家不要习惯性的去责备人的利己心,如果一种利己的行为,它导致了一个很恶劣的结果,往往不是因为利己心出问题了,而是因为游戏规则出问题了。
        要改变的不是利己心,要改变的是我们的游戏规则和制度。

        \clearpage
        \section{结束语}
        人终有一死,我们会被他人取而代之,因此我们不能传承个人的经历,因为它将随我们而去,我们能够留下的只是制度,人类在生物演变过程中取得现如今的地位,
        靠的是符号和合作,我们才能传承数千年前的人类智慧,为此我们不惜用人生1/3的时间集中用来学习前人的经验,基于专业化分工基础上的人类合作秩序的不断拓展,
        是人类社会快速发展的最大源泉。\par

        人生是永不停歇的博弈过程,博弈的精髓不是要通过阴谋诡计,通过暴力把对方骗了,甚至给灭了,而是要通过共同努力的去建立起更好的游戏规则和合作关系,
        变有限游戏为无限游戏,实现彼此合作共赢的目标。单赢是不长久的,你不可能永远赢,只有双赢才长久,是可持续的,博弈的最高境界其实是双赢。



    \end{CJK*}
\end{document}
