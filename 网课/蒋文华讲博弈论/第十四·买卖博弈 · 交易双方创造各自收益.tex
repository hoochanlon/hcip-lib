%!TEX program = xelatex
%!TEX TS-program = xelatex
%!TEX encoding = UTF-8 Unicode

\documentclass[12pt, a4paper]{article}
\usepackage{CJKutf8}
\usepackage{graphicx}
\usepackage{subfigure}
\usepackage{listings}
\usepackage[colorlinks,linkcolor=blue]{hyperref}
\usepackage{ulem}
\usepackage{xcolor}
\usepackage{caption2}
\usepackage{amssymb}
\setlength{\parskip}{1em}


\title{第十四·买卖博弈 · 交易双方创造各自收益}
\author{hoochanlon}
\date{\today}


\begin{document}
	\begin{CJK*}{UTF8}{gbsn}
		\maketitle

        \clearpage
        \section{交易双方创造各自收益}
        \begin{flushleft}
        买卖双方的博弈,有三种方式:
        \begin{enumerate}
            \item 独裁者博弈:强买强卖
            \item 最后通牒博弈:爱玩玩,不玩滚
            \item 讨价还价博弈:讨价还价
        \end{enumerate}
        讨价还价博弈相对于最后通牒博弈而言,还价的一方拥有了拒绝的权利,还拥有还价的权利。
        \subsection{讨价还价概念}
        讨价还价也称议价或谈判,主要指参与者通过协商方式,解决利益的分配问题。称讨价还价时,主要强调其动作或过程,称谈判时,则强调状态或结果。\par

        谢林的论文《试论讨价还价》,首先发表在1956年的《美国经济评论》上面,后被收编入《冲突的策略》的第二章中,通过对讨价还价现象的分析,
        谢林得出意想不到的结论~~—在讨价还价的过程中,弱势的一方通常会成为强者。 买方的谈判关键,让对方相信你不会再让步了。\par

        \subsection{贴现因子}
        \subsubsection{分蛋糕模型}
        1982年鲁宾斯坦对基本的、无限期的完全信息的讨价还价过程进行了分析,并且构建了一个基于他命名的“鲁宾斯坦”模型,即分蛋糕\footnote{\href{http://www.baidu.com/}{百度百科~-~鲁宾斯坦模型}}:
        \begin{quote}
            {\small
            在这个模型里,两个参与人分割一块蛋糕,参与人1先出价,参与人2可以选择接受或拒绝。如果参与人2接受,则博弈结束,蛋糕按参与人1的方案分配;
            如果参与人2拒绝,他将还价,参与人1可以接受或拒绝;如果参与人1接受,博弈结束,蛋糕按参与人2的方案分配;
            如果参与人1拒绝,他再出价;如此一直下去,直到一个参与人的出价被另一个参与人接受为止。
            因此,这属于一个无限期完美信息博弈,参与人1在时期1,3,5,⋯ 出价,参与人2在时期2,4,6,⋯出价。}
        \end{quote}
        更为明显的例子,甲乙双方开价,十万各自分成,四六,五五及其他。如果是说,先给四万,之后再给五万,这就涉及到一个跨期选择的机会成本问题。
        比方说,我现在给你一百块,一个星期后给你一百一十块,问你会选哪一种?

        \subsubsection{贴现因子概念}

        眼前利益与长远利益,一百一增值的十块钱,转换成利率一星期后增值10\%。我们怎样把以后的钱和成现在的钱进行数量上的比较。
        任何东西我们要进行比较的话,一定是相同性质的才行。要比较也需把以后的钱折算成现在的钱,而这个就涉及到了“贴现因子”。
        贴现因子也称折现系数,折现参数,也可以理解为超市的折扣。把未来的收益,折算成眼前的收益,贴现因子越高,越看重长远利益。贴现因子反应了眼前利益与长远利益的权衡。

        \subsubsection{酸葡萄和甜葡萄}
        \begin{itemize}
            \item 先吃酸的,看中未来利益,先苦后甜型。
            \item 先吃甜的,看重眼前利益,先甜后苦型。
            \item 抓到哪颗,随机选择,随遇而安型。
        \end{itemize}
        贴现因子由参与者耐心决定,实际上就是参与者心理和经济承受能力。

        \subsection{讨价还价模型均衡求解}
        假设:参与者1所得的份额是X,参与者2所的份额1-x;参与者1的贴现因子是$\delta_1$,参与者2的贴现因子是$\delta_2$。 \par
        $X_i$和$1-X_i$分别是时期i时,参与者1和参与者2各自所得份额。
        \begin{itemize}
            \item 参与者1支付的贴现值为:$W_1=\delta^{t-1}_1X_t$
            \item 参与者2支付的贴现值为:$W_2=\delta^{t-1}_2(1-X_t)$
            \item 参与者1获得的份额为:$X^*=(1-\delta_2)/(1-\delta_1 \times \delta_2)$
            \item 参与者2获得的份额为:$1-X^*$ \par
        \end{itemize}

        \subsubsection{从均衡解得到的推论}
        \begin{itemize}
            \item 当$\delta_1=\delta_2$时,$X^*=1/(1+\delta)>1/2$ 先开价处于有利地位,即先动优势。
            \item 当$\delta_1=1$时,$X^*=1$ ;当$\delta_2=1$时,$X^*=0$ ;当$\delta_2=1$时,$X^*=0$ 第一人有无穷的大耐心,他可以获得所有收益。
            \item 当$\delta_1=\delta_2=1$时,$X^*=0/0$,结果无解。
        \end{itemize}
        随着$\delta_1$的增加,$X_1^*$不断增加;随着$\delta_2$的增加,$X_1^*$不断减少。

        \subsubsection{生活中的例子}
        租赁关系:房屋买卖,是有利于卖方的,因为卖方可以等待,而买方不能等待,需要入住安置。\\
        买卖关系:“房东出国,急于卖房”、“房东资金周转,急于卖房”,因此会让我们认为购买价格更低。(言语上的信息)

        \subsubsection{影响贴现因子分析的因素分析}
        \begin{itemize}
            \item 预期寿命:贴现因子小,竭泽而渔;贴现因子大,休养生息。
            \item 未来收益的确定性:眼前的利益有确定性,未来的利益有不确定性。(一鸟在手,胜似两鸟在林;盛世古董,乱世黄金。)
            \item 财富存量:财富越多越看重长远利益。财富太少,先重温饱。(远水解不了近渴;人穷志短。)
            \item 知识或文化程度:知识文化水平越高,贴现因子越大。知识水平和文化程度有助于增强人们对未来收益的想象力。从而使人们更愿意放弃眼前利益,这种信仰也是对长远未来的想象力。
        \end{itemize}
        \subsection{征地拆迁的讨价还价}
        农用地转化为建设用地,土地增值的部分,被视作土地红利。围绕这新增的蛋糕,征地拆迁一方和被征地拆迁一方往往会有经济利益冲突。
        在讨价还价的博弈中,我们需要对蛋糕的大小有共识,然而在拆迁过程中,对不动产的评估方法会不一样:
        \begin{itemize}
        \item 政府:成本法计价,当时花费的成本,加适当补偿。(升值部分由国家政府和公众共享)
        \item 拆迁户:市场的比较法计价,市场价格。(升值部分由个人享有)
        \end{itemize}
        平等主体之间的讨价还价,无论结果如何,都不可能损害任何一方的利益,双方都能接受最后的谈判结果。一旦双方的地位是不平等的,那么强制行为
        就必然会发生。当双方都无法解决,怎么引入公正的第三方机制就变得非常重要了。

        \end{flushleft}
    \end{CJK*}
\end{document}
